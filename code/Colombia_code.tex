% Options for packages loaded elsewhere
\PassOptionsToPackage{unicode}{hyperref}
\PassOptionsToPackage{hyphens}{url}
%
\documentclass[
]{article}
\usepackage{amsmath,amssymb}
\usepackage{iftex}
\ifPDFTeX
  \usepackage[T1]{fontenc}
  \usepackage[utf8]{inputenc}
  \usepackage{textcomp} % provide euro and other symbols
\else % if luatex or xetex
  \usepackage{unicode-math} % this also loads fontspec
  \defaultfontfeatures{Scale=MatchLowercase}
  \defaultfontfeatures[\rmfamily]{Ligatures=TeX,Scale=1}
\fi
\usepackage{lmodern}
\ifPDFTeX\else
  % xetex/luatex font selection
\fi
% Use upquote if available, for straight quotes in verbatim environments
\IfFileExists{upquote.sty}{\usepackage{upquote}}{}
\IfFileExists{microtype.sty}{% use microtype if available
  \usepackage[]{microtype}
  \UseMicrotypeSet[protrusion]{basicmath} % disable protrusion for tt fonts
}{}
\makeatletter
\@ifundefined{KOMAClassName}{% if non-KOMA class
  \IfFileExists{parskip.sty}{%
    \usepackage{parskip}
  }{% else
    \setlength{\parindent}{0pt}
    \setlength{\parskip}{6pt plus 2pt minus 1pt}}
}{% if KOMA class
  \KOMAoptions{parskip=half}}
\makeatother
\usepackage{xcolor}
\usepackage[margin=1in]{geometry}
\usepackage{color}
\usepackage{fancyvrb}
\newcommand{\VerbBar}{|}
\newcommand{\VERB}{\Verb[commandchars=\\\{\}]}
\DefineVerbatimEnvironment{Highlighting}{Verbatim}{commandchars=\\\{\}}
% Add ',fontsize=\small' for more characters per line
\usepackage{framed}
\definecolor{shadecolor}{RGB}{248,248,248}
\newenvironment{Shaded}{\begin{snugshade}}{\end{snugshade}}
\newcommand{\AlertTok}[1]{\textcolor[rgb]{0.94,0.16,0.16}{#1}}
\newcommand{\AnnotationTok}[1]{\textcolor[rgb]{0.56,0.35,0.01}{\textbf{\textit{#1}}}}
\newcommand{\AttributeTok}[1]{\textcolor[rgb]{0.13,0.29,0.53}{#1}}
\newcommand{\BaseNTok}[1]{\textcolor[rgb]{0.00,0.00,0.81}{#1}}
\newcommand{\BuiltInTok}[1]{#1}
\newcommand{\CharTok}[1]{\textcolor[rgb]{0.31,0.60,0.02}{#1}}
\newcommand{\CommentTok}[1]{\textcolor[rgb]{0.56,0.35,0.01}{\textit{#1}}}
\newcommand{\CommentVarTok}[1]{\textcolor[rgb]{0.56,0.35,0.01}{\textbf{\textit{#1}}}}
\newcommand{\ConstantTok}[1]{\textcolor[rgb]{0.56,0.35,0.01}{#1}}
\newcommand{\ControlFlowTok}[1]{\textcolor[rgb]{0.13,0.29,0.53}{\textbf{#1}}}
\newcommand{\DataTypeTok}[1]{\textcolor[rgb]{0.13,0.29,0.53}{#1}}
\newcommand{\DecValTok}[1]{\textcolor[rgb]{0.00,0.00,0.81}{#1}}
\newcommand{\DocumentationTok}[1]{\textcolor[rgb]{0.56,0.35,0.01}{\textbf{\textit{#1}}}}
\newcommand{\ErrorTok}[1]{\textcolor[rgb]{0.64,0.00,0.00}{\textbf{#1}}}
\newcommand{\ExtensionTok}[1]{#1}
\newcommand{\FloatTok}[1]{\textcolor[rgb]{0.00,0.00,0.81}{#1}}
\newcommand{\FunctionTok}[1]{\textcolor[rgb]{0.13,0.29,0.53}{\textbf{#1}}}
\newcommand{\ImportTok}[1]{#1}
\newcommand{\InformationTok}[1]{\textcolor[rgb]{0.56,0.35,0.01}{\textbf{\textit{#1}}}}
\newcommand{\KeywordTok}[1]{\textcolor[rgb]{0.13,0.29,0.53}{\textbf{#1}}}
\newcommand{\NormalTok}[1]{#1}
\newcommand{\OperatorTok}[1]{\textcolor[rgb]{0.81,0.36,0.00}{\textbf{#1}}}
\newcommand{\OtherTok}[1]{\textcolor[rgb]{0.56,0.35,0.01}{#1}}
\newcommand{\PreprocessorTok}[1]{\textcolor[rgb]{0.56,0.35,0.01}{\textit{#1}}}
\newcommand{\RegionMarkerTok}[1]{#1}
\newcommand{\SpecialCharTok}[1]{\textcolor[rgb]{0.81,0.36,0.00}{\textbf{#1}}}
\newcommand{\SpecialStringTok}[1]{\textcolor[rgb]{0.31,0.60,0.02}{#1}}
\newcommand{\StringTok}[1]{\textcolor[rgb]{0.31,0.60,0.02}{#1}}
\newcommand{\VariableTok}[1]{\textcolor[rgb]{0.00,0.00,0.00}{#1}}
\newcommand{\VerbatimStringTok}[1]{\textcolor[rgb]{0.31,0.60,0.02}{#1}}
\newcommand{\WarningTok}[1]{\textcolor[rgb]{0.56,0.35,0.01}{\textbf{\textit{#1}}}}
\usepackage{graphicx}
\makeatletter
\def\maxwidth{\ifdim\Gin@nat@width>\linewidth\linewidth\else\Gin@nat@width\fi}
\def\maxheight{\ifdim\Gin@nat@height>\textheight\textheight\else\Gin@nat@height\fi}
\makeatother
% Scale images if necessary, so that they will not overflow the page
% margins by default, and it is still possible to overwrite the defaults
% using explicit options in \includegraphics[width, height, ...]{}
\setkeys{Gin}{width=\maxwidth,height=\maxheight,keepaspectratio}
% Set default figure placement to htbp
\makeatletter
\def\fps@figure{htbp}
\makeatother
\setlength{\emergencystretch}{3em} % prevent overfull lines
\providecommand{\tightlist}{%
  \setlength{\itemsep}{0pt}\setlength{\parskip}{0pt}}
\setcounter{secnumdepth}{-\maxdimen} % remove section numbering
\usepackage{booktabs}
\usepackage{longtable}
\usepackage{array}
\usepackage{multirow}
\usepackage{wrapfig}
\usepackage{float}
\usepackage{colortbl}
\usepackage{pdflscape}
\usepackage{tabu}
\usepackage{threeparttable}
\usepackage{threeparttablex}
\usepackage[normalem]{ulem}
\usepackage{makecell}
\usepackage{xcolor}
\ifLuaTeX
  \usepackage{selnolig}  % disable illegal ligatures
\fi
\usepackage{bookmark}
\IfFileExists{xurl.sty}{\usepackage{xurl}}{} % add URL line breaks if available
\urlstyle{same}
\hypersetup{
  pdftitle={Colombia\_code},
  hidelinks,
  pdfcreator={LaTeX via pandoc}}

\title{Colombia\_code}
\author{}
\date{\vspace{-2.5em}2025-04-02}

\begin{document}
\maketitle

\begin{Shaded}
\begin{Highlighting}[]
\FunctionTok{library}\NormalTok{(readxl)}
\FunctionTok{library}\NormalTok{(openxlsx)}
\FunctionTok{library}\NormalTok{(dplyr)}
\FunctionTok{library}\NormalTok{(lubridate)}
\FunctionTok{library}\NormalTok{(ggplot2)}
\FunctionTok{library}\NormalTok{(forecast)}
\FunctionTok{library}\NormalTok{(Kendall)}
\FunctionTok{library}\NormalTok{(tseries)}
\FunctionTok{library}\NormalTok{(outliers)}
\FunctionTok{library}\NormalTok{(tidyverse)}
\FunctionTok{library}\NormalTok{(smooth)}
\FunctionTok{library}\NormalTok{(zoo)}
\FunctionTok{library}\NormalTok{(kableExtra)}
\FunctionTok{library}\NormalTok{(writexl)}
\FunctionTok{library}\NormalTok{(tsibble)}
\FunctionTok{library}\NormalTok{(fable)}
\FunctionTok{library}\NormalTok{(fable.prophet)}
\end{Highlighting}
\end{Shaded}

\subsection{Importing Data}\label{importing-data}

\begin{Shaded}
\begin{Highlighting}[]
\CommentTok{\# importing data}
\NormalTok{Colombia\_pre\_uncleaned }\OtherTok{\textless{}{-}} \FunctionTok{read\_excel}\NormalTok{(}\StringTok{"../data/Colombia\_precipitation.xlsx"}\NormalTok{)  }\CommentTok{\# I changed the to the relative address }

\FunctionTok{colSums}\NormalTok{(}\FunctionTok{is.na}\NormalTok{(Colombia\_pre\_uncleaned))}
\end{Highlighting}
\end{Shaded}

\begin{verbatim}
## YEAR  JAN  FEB  MAR  APR  MAY  JUN  JUL  AUG  SEP  OCT  NOV  DEC  MAM  JJA  SON 
##    0    0    0    0    0    0    0    0    0    0    0    0    0    0    0    0 
##  DJF  ANN 
##    1    0
\end{verbatim}

\begin{Shaded}
\begin{Highlighting}[]
\NormalTok{Colombia\_pre\_uncleaned}\SpecialCharTok{$}\NormalTok{DJF[}\FunctionTok{is.na}\NormalTok{(Colombia\_pre\_uncleaned}\SpecialCharTok{$}\NormalTok{DJF)] }\OtherTok{\textless{}{-}} \FunctionTok{mean}\NormalTok{(Colombia\_pre\_uncleaned}\SpecialCharTok{$}\NormalTok{DJF,}
    \AttributeTok{na.rm =} \ConstantTok{TRUE}\NormalTok{)}
\FunctionTok{colSums}\NormalTok{(}\FunctionTok{is.na}\NormalTok{(Colombia\_pre\_uncleaned))}
\end{Highlighting}
\end{Shaded}

\begin{verbatim}
## YEAR  JAN  FEB  MAR  APR  MAY  JUN  JUL  AUG  SEP  OCT  NOV  DEC  MAM  JJA  SON 
##    0    0    0    0    0    0    0    0    0    0    0    0    0    0    0    0 
##  DJF  ANN 
##    0    0
\end{verbatim}

\begin{Shaded}
\begin{Highlighting}[]
\FunctionTok{write.csv}\NormalTok{(Colombia\_pre\_uncleaned, }\StringTok{"../data/processed\_colombia\_precipitation.csv"}\NormalTok{,}
    \AttributeTok{row.names =} \ConstantTok{FALSE}\NormalTok{)}

\NormalTok{colombia\_pre }\OtherTok{\textless{}{-}} \FunctionTok{read\_csv}\NormalTok{(}\StringTok{"../data/processed\_colombia\_precipitation.csv"}\NormalTok{)}
\end{Highlighting}
\end{Shaded}

\begin{verbatim}
## Rows: 124 Columns: 18
## -- Column specification --------------------------------------------------------
## Delimiter: ","
## dbl (18): YEAR, JAN, FEB, MAR, APR, MAY, JUN, JUL, AUG, SEP, OCT, NOV, DEC, ...
## 
## i Use `spec()` to retrieve the full column specification for this data.
## i Specify the column types or set `show_col_types = FALSE` to quiet this message.
\end{verbatim}

\begin{Shaded}
\begin{Highlighting}[]
\FunctionTok{head}\NormalTok{(colombia\_pre)}
\end{Highlighting}
\end{Shaded}

\begin{verbatim}
## # A tibble: 6 x 18
##    YEAR   JAN   FEB   MAR   APR   MAY   JUN   JUL   AUG   SEP   OCT   NOV   DEC
##   <dbl> <dbl> <dbl> <dbl> <dbl> <dbl> <dbl> <dbl> <dbl> <dbl> <dbl> <dbl> <dbl>
## 1  1901  90.1 102.   151.  209.  278.  256.  312.  329.  230.  288.  236.  128.
## 2  1902 138.   96.9  188.  230.  261.  254.  227.  226.  225   233.  198.  119.
## 3  1903  99.7  98.3  136.  230.  278.  336.  226.  316.  223.  226.  215.  148.
## 4  1904 111   112.   189.  271.  304   250.  264.  228.  219.  262.  169.  114.
## 5  1905 112.   87.7  142.  239.  286.  273.  248.  211.  273   257.  231.  163.
## 6  1906  90   108.   147.  271.  297.  309.  287.  238.  195.  252   192.  130.
## # i 5 more variables: MAM <dbl>, JJA <dbl>, SON <dbl>, DJF <dbl>, ANN <dbl>
\end{verbatim}

\begin{Shaded}
\begin{Highlighting}[]
\NormalTok{nvar }\OtherTok{\textless{}{-}} \FunctionTok{ncol}\NormalTok{(colombia\_pre) }\SpecialCharTok{{-}} \DecValTok{1}
\NormalTok{nobs }\OtherTok{\textless{}{-}} \FunctionTok{nrow}\NormalTok{(colombia\_pre)}
\end{Highlighting}
\end{Shaded}

\subsection{Data Cleaning \& TS
Creation}\label{data-cleaning-ts-creation}

\begin{Shaded}
\begin{Highlighting}[]
\NormalTok{monthly\_data }\OtherTok{\textless{}{-}}\NormalTok{ colombia\_pre }\SpecialCharTok{\%\textgreater{}\%}
    \FunctionTok{select}\NormalTok{(YEAR, JAN, FEB, MAR, APR, MAY, JUN, JUL, AUG, SEP,}
\NormalTok{        OCT, NOV, DEC)}

\NormalTok{monthly\_data\_long }\OtherTok{\textless{}{-}}\NormalTok{ monthly\_data }\SpecialCharTok{\%\textgreater{}\%}
    \FunctionTok{pivot\_longer}\NormalTok{(}\AttributeTok{cols =}\NormalTok{ JAN}\SpecialCharTok{:}\NormalTok{DEC, }\AttributeTok{names\_to =} \StringTok{"Month"}\NormalTok{, }\AttributeTok{values\_to =} \StringTok{"Precipitation"}\NormalTok{)}

\NormalTok{monthly\_data\_long }\OtherTok{\textless{}{-}}\NormalTok{ monthly\_data\_long }\SpecialCharTok{\%\textgreater{}\%}
    \FunctionTok{mutate}\NormalTok{(}\AttributeTok{Month\_Num =} \FunctionTok{match}\NormalTok{(}\FunctionTok{tolower}\NormalTok{(Month), }\FunctionTok{tolower}\NormalTok{(month.abb)),}
        \AttributeTok{Date =} \FunctionTok{as.Date}\NormalTok{(}\FunctionTok{paste}\NormalTok{(YEAR, Month\_Num, }\StringTok{"01"}\NormalTok{, }\AttributeTok{sep =} \StringTok{"{-}"}\NormalTok{)),}
        \AttributeTok{.before =} \DecValTok{1}\NormalTok{) }\SpecialCharTok{\%\textgreater{}\%}
    \FunctionTok{arrange}\NormalTok{(Date)}

\FunctionTok{write.csv}\NormalTok{(monthly\_data\_long, }\StringTok{"../data/colombiamonthly\_data\_long.csv"}\NormalTok{,}
    \AttributeTok{row.names =} \ConstantTok{FALSE}\NormalTok{)}

\NormalTok{ts\_monthly }\OtherTok{\textless{}{-}} \FunctionTok{ts}\NormalTok{(monthly\_data\_long}\SpecialCharTok{$}\NormalTok{Precipitation, }\AttributeTok{start =} \FunctionTok{c}\NormalTok{(}\DecValTok{1901}\NormalTok{,}
    \DecValTok{1}\NormalTok{), }\AttributeTok{frequency =} \DecValTok{12}\NormalTok{)}
\end{Highlighting}
\end{Shaded}

\subsection{Initial Plots}\label{initial-plots}

\begin{Shaded}
\begin{Highlighting}[]
\FunctionTok{autoplot}\NormalTok{(ts\_monthly) }\SpecialCharTok{+} \FunctionTok{ggtitle}\NormalTok{(}\StringTok{"Monthly Precipitation Time Series (Colombia)"}\NormalTok{) }\SpecialCharTok{+}
    \FunctionTok{xlab}\NormalTok{(}\StringTok{"Year"}\NormalTok{) }\SpecialCharTok{+} \FunctionTok{ylab}\NormalTok{(}\StringTok{"Precipitation (mm)"}\NormalTok{) }\SpecialCharTok{+} \FunctionTok{theme\_minimal}\NormalTok{()}
\end{Highlighting}
\end{Shaded}

\includegraphics{Colombia_code_files/figure-latex/unnamed-chunk-3-1.pdf}

\begin{Shaded}
\begin{Highlighting}[]
\FunctionTok{par}\NormalTok{(}\AttributeTok{mfrow =} \FunctionTok{c}\NormalTok{(}\DecValTok{1}\NormalTok{, }\DecValTok{2}\NormalTok{))}
\FunctionTok{acf}\NormalTok{(ts\_monthly, }\AttributeTok{lag.max =} \DecValTok{48}\NormalTok{, }\AttributeTok{main =} \StringTok{"ACF (Colombia)"}\NormalTok{)}
\FunctionTok{pacf}\NormalTok{(ts\_monthly, }\AttributeTok{lag.max =} \DecValTok{48}\NormalTok{, }\AttributeTok{main =} \StringTok{"PACF (Colombia)"}\NormalTok{)}
\end{Highlighting}
\end{Shaded}

\includegraphics{Colombia_code_files/figure-latex/unnamed-chunk-3-2.pdf}

\begin{Shaded}
\begin{Highlighting}[]
\FunctionTok{par}\NormalTok{(}\AttributeTok{mfrow =} \FunctionTok{c}\NormalTok{(}\DecValTok{1}\NormalTok{, }\DecValTok{1}\NormalTok{))}
\end{Highlighting}
\end{Shaded}

\subsection{Testing and Training}\label{testing-and-training}

\begin{Shaded}
\begin{Highlighting}[]
\NormalTok{train\_precipitation }\OtherTok{\textless{}{-}} \FunctionTok{window}\NormalTok{(ts\_monthly, }\AttributeTok{end =} \FunctionTok{c}\NormalTok{(}\DecValTok{2020}\NormalTok{, }\DecValTok{12}\NormalTok{))}

\NormalTok{h }\OtherTok{\textless{}{-}} \DecValTok{240}

\NormalTok{total\_months }\OtherTok{\textless{}{-}} \FunctionTok{length}\NormalTok{(ts\_monthly)}

\NormalTok{test\_precipitation }\OtherTok{\textless{}{-}} \FunctionTok{window}\NormalTok{(ts\_monthly, }\AttributeTok{start =} \FunctionTok{c}\NormalTok{(}\DecValTok{2021}\NormalTok{, }\DecValTok{1}\NormalTok{),}
    \AttributeTok{end =} \FunctionTok{c}\NormalTok{(}\DecValTok{2024}\NormalTok{, }\DecValTok{12}\NormalTok{))}

\FunctionTok{autoplot}\NormalTok{(train\_precipitation)}
\end{Highlighting}
\end{Shaded}

\includegraphics{Colombia_code_files/figure-latex/unnamed-chunk-4-1.pdf}

\begin{Shaded}
\begin{Highlighting}[]
\FunctionTok{autoplot}\NormalTok{(test\_precipitation)}
\end{Highlighting}
\end{Shaded}

\includegraphics{Colombia_code_files/figure-latex/unnamed-chunk-4-2.pdf}

\subsection{Decomposing Time Series}\label{decomposing-time-series}

\begin{Shaded}
\begin{Highlighting}[]
\CommentTok{\# Decompose}
\NormalTok{decompose\_monthly }\OtherTok{\textless{}{-}} \FunctionTok{decompose}\NormalTok{(ts\_monthly, }\StringTok{"additive"}\NormalTok{)}
\FunctionTok{plot}\NormalTok{(decompose\_monthly)}
\end{Highlighting}
\end{Shaded}

\includegraphics{Colombia_code_files/figure-latex/unnamed-chunk-5-1.pdf}

\begin{Shaded}
\begin{Highlighting}[]
\CommentTok{\# Creating non{-}seasonal time series}
\NormalTok{deseasonal\_monthly }\OtherTok{\textless{}{-}} \FunctionTok{seasadj}\NormalTok{(decompose\_monthly)}
\end{Highlighting}
\end{Shaded}

\subsection{ARIMA}\label{arima}

\begin{Shaded}
\begin{Highlighting}[]
\NormalTok{fit\_arima }\OtherTok{\textless{}{-}} \FunctionTok{auto.arima}\NormalTok{(train\_precipitation)}
\NormalTok{pre\_arima }\OtherTok{\textless{}{-}} \FunctionTok{forecast}\NormalTok{(fit\_arima, }\AttributeTok{h =} \DecValTok{360}\NormalTok{)}

\FunctionTok{plot}\NormalTok{(pre\_arima, }\AttributeTok{main =} \StringTok{"ARIMA Forecast (Colombia)"}\NormalTok{)}
\end{Highlighting}
\end{Shaded}

\includegraphics{Colombia_code_files/figure-latex/unnamed-chunk-6-1.pdf}

\begin{Shaded}
\begin{Highlighting}[]
\CommentTok{\# Plot model + observed data}
\FunctionTok{autoplot}\NormalTok{(ts\_monthly) }\SpecialCharTok{+} \FunctionTok{autolayer}\NormalTok{(pre\_arima, }\AttributeTok{series =} \StringTok{"ARIMA"}\NormalTok{,}
    \AttributeTok{PI =} \ConstantTok{FALSE}\NormalTok{) }\SpecialCharTok{+} \FunctionTok{ylab}\NormalTok{(}\StringTok{"Precipitation (mm)"}\NormalTok{)}
\end{Highlighting}
\end{Shaded}

\includegraphics{Colombia_code_files/figure-latex/unnamed-chunk-6-2.pdf}

\subsection{STL + ETS}\label{stl-ets}

\begin{Shaded}
\begin{Highlighting}[]
\NormalTok{pre\_stl\_ets }\OtherTok{\textless{}{-}} \FunctionTok{stlf}\NormalTok{(train\_precipitation, }\AttributeTok{h =} \DecValTok{360}\NormalTok{, }\AttributeTok{method =} \StringTok{"ets"}\NormalTok{)}
\FunctionTok{plot}\NormalTok{(pre\_stl\_ets, }\AttributeTok{main =} \StringTok{"STL + ETS Forecast (Colombia)"}\NormalTok{)}
\end{Highlighting}
\end{Shaded}

\includegraphics{Colombia_code_files/figure-latex/unnamed-chunk-7-1.pdf}

\begin{Shaded}
\begin{Highlighting}[]
\CommentTok{\# Plot model + observed data}
\FunctionTok{autoplot}\NormalTok{(ts\_monthly) }\SpecialCharTok{+} \FunctionTok{autolayer}\NormalTok{(pre\_stl\_ets, }\AttributeTok{series =} \StringTok{"STL + ETS"}\NormalTok{,}
    \AttributeTok{PI =} \ConstantTok{FALSE}\NormalTok{) }\SpecialCharTok{+} \FunctionTok{ylab}\NormalTok{(}\StringTok{"Precipitation (mm)"}\NormalTok{)}
\end{Highlighting}
\end{Shaded}

\includegraphics{Colombia_code_files/figure-latex/unnamed-chunk-7-2.pdf}

\subsection{ARIMA + Fourier terms}\label{arima-fourier-terms}

\begin{Shaded}
\begin{Highlighting}[]
\NormalTok{K }\OtherTok{\textless{}{-}} \DecValTok{6}
\NormalTok{fourier\_train }\OtherTok{\textless{}{-}} \FunctionTok{fourier}\NormalTok{(train\_precipitation, }\AttributeTok{K =}\NormalTok{ K)}
\NormalTok{fourier\_future }\OtherTok{\textless{}{-}} \FunctionTok{fourier}\NormalTok{(train\_precipitation, }\AttributeTok{K =}\NormalTok{ K, }\AttributeTok{h =} \DecValTok{360}\NormalTok{)}
\NormalTok{fit\_fourier }\OtherTok{\textless{}{-}} \FunctionTok{auto.arima}\NormalTok{(train\_precipitation, }\AttributeTok{xreg =}\NormalTok{ fourier\_train,}
    \AttributeTok{seasonal =} \ConstantTok{FALSE}\NormalTok{)}
\NormalTok{pre\_fourier }\OtherTok{\textless{}{-}} \FunctionTok{forecast}\NormalTok{(fit\_fourier, }\AttributeTok{xreg =}\NormalTok{ fourier\_future, }\AttributeTok{h =} \DecValTok{360}\NormalTok{)}

\FunctionTok{plot}\NormalTok{(pre\_fourier, }\AttributeTok{main =} \StringTok{"ARIMA + Fourier Forecast (Colombia)"}\NormalTok{)}
\end{Highlighting}
\end{Shaded}

\includegraphics{Colombia_code_files/figure-latex/unnamed-chunk-8-1.pdf}

\begin{Shaded}
\begin{Highlighting}[]
\CommentTok{\# Plot model + observed data}
\FunctionTok{autoplot}\NormalTok{(ts\_monthly) }\SpecialCharTok{+} \FunctionTok{autolayer}\NormalTok{(pre\_fourier, }\AttributeTok{series =} \StringTok{"ARIMA + Fourier terms"}\NormalTok{,}
    \AttributeTok{PI =} \ConstantTok{FALSE}\NormalTok{) }\SpecialCharTok{+} \FunctionTok{ylab}\NormalTok{(}\StringTok{"Precipitation (mm)"}\NormalTok{)}
\end{Highlighting}
\end{Shaded}

\includegraphics{Colombia_code_files/figure-latex/unnamed-chunk-8-2.pdf}

\subsection{TBATS}\label{tbats}

\begin{Shaded}
\begin{Highlighting}[]
\NormalTok{fit\_tbats }\OtherTok{\textless{}{-}} \FunctionTok{tbats}\NormalTok{(train\_precipitation)}
\NormalTok{pre\_tbats }\OtherTok{\textless{}{-}} \FunctionTok{forecast}\NormalTok{(fit\_tbats, }\AttributeTok{h =} \DecValTok{360}\NormalTok{)}
\FunctionTok{plot}\NormalTok{(pre\_tbats, }\AttributeTok{main =} \StringTok{"TBATS Forecast (Colombia)"}\NormalTok{)}
\end{Highlighting}
\end{Shaded}

\includegraphics{Colombia_code_files/figure-latex/unnamed-chunk-9-1.pdf}

\begin{Shaded}
\begin{Highlighting}[]
\CommentTok{\# Plot model + observed data}
\FunctionTok{autoplot}\NormalTok{(ts\_monthly) }\SpecialCharTok{+} \FunctionTok{autolayer}\NormalTok{(pre\_tbats, }\AttributeTok{series =} \StringTok{"TBATS"}\NormalTok{,}
    \AttributeTok{PI =} \ConstantTok{FALSE}\NormalTok{) }\SpecialCharTok{+} \FunctionTok{ylab}\NormalTok{(}\StringTok{"Precipitation (mm)"}\NormalTok{)}
\end{Highlighting}
\end{Shaded}

\includegraphics{Colombia_code_files/figure-latex/unnamed-chunk-9-2.pdf}

\subsection{Neural Network}\label{neural-network}

\begin{Shaded}
\begin{Highlighting}[]
\NormalTok{fit\_nnetar }\OtherTok{\textless{}{-}} \FunctionTok{nnetar}\NormalTok{(train\_precipitation)}
\NormalTok{pre\_nnetar }\OtherTok{\textless{}{-}} \FunctionTok{forecast}\NormalTok{(fit\_nnetar, }\AttributeTok{h =} \DecValTok{360}\NormalTok{)}
\FunctionTok{plot}\NormalTok{(pre\_nnetar, }\AttributeTok{main =} \StringTok{"NNETAR Forecast (Colombia)"}\NormalTok{)}
\end{Highlighting}
\end{Shaded}

\includegraphics{Colombia_code_files/figure-latex/unnamed-chunk-10-1.pdf}

\begin{Shaded}
\begin{Highlighting}[]
\CommentTok{\# Plot model + observed data}
\FunctionTok{autoplot}\NormalTok{(ts\_monthly) }\SpecialCharTok{+} \FunctionTok{autolayer}\NormalTok{(pre\_nnetar, }\AttributeTok{series =} \StringTok{"Neural Network"}\NormalTok{,}
    \AttributeTok{PI =} \ConstantTok{FALSE}\NormalTok{) }\SpecialCharTok{+} \FunctionTok{ylab}\NormalTok{(}\StringTok{"Precipitation (mm)"}\NormalTok{)}
\end{Highlighting}
\end{Shaded}

\includegraphics{Colombia_code_files/figure-latex/unnamed-chunk-10-2.pdf}

\subsection{Scores}\label{scores}

\begin{Shaded}
\begin{Highlighting}[]
\CommentTok{\# Model 1: ARIMA}
\NormalTok{ARIMA\_scores }\OtherTok{\textless{}{-}} \FunctionTok{accuracy}\NormalTok{(pre\_arima}\SpecialCharTok{$}\NormalTok{mean, test\_precipitation)}

\CommentTok{\# Model 2: STL + ETS}
\NormalTok{ETS\_scores }\OtherTok{\textless{}{-}} \FunctionTok{accuracy}\NormalTok{(pre\_stl\_ets}\SpecialCharTok{$}\NormalTok{mean, test\_precipitation)}

\CommentTok{\# Model 3: ARIMA + Fourier}
\NormalTok{ARIMA\_scores }\OtherTok{\textless{}{-}} \FunctionTok{accuracy}\NormalTok{(pre\_fourier}\SpecialCharTok{$}\NormalTok{mean, test\_precipitation)}

\CommentTok{\# Model 4: TBATS}
\NormalTok{TBATS\_scores }\OtherTok{\textless{}{-}} \FunctionTok{accuracy}\NormalTok{(pre\_tbats}\SpecialCharTok{$}\NormalTok{mean, test\_precipitation)}

\CommentTok{\# Model 5: Neural Network}
\NormalTok{NN\_scores }\OtherTok{\textless{}{-}} \FunctionTok{accuracy}\NormalTok{(pre\_nnetar}\SpecialCharTok{$}\NormalTok{mean, test\_precipitation)}
\end{Highlighting}
\end{Shaded}

\begin{Shaded}
\begin{Highlighting}[]
\CommentTok{\# create data frame}
\NormalTok{scores }\OtherTok{\textless{}{-}} \FunctionTok{as.data.frame}\NormalTok{(}\FunctionTok{rbind}\NormalTok{(ARIMA\_scores, ETS\_scores, ARIMA\_scores,}
\NormalTok{    TBATS\_scores, NN\_scores))}
\FunctionTok{row.names}\NormalTok{(scores) }\OtherTok{\textless{}{-}} \FunctionTok{c}\NormalTok{(}\StringTok{"ARIMA"}\NormalTok{, }\StringTok{"STL+ETS"}\NormalTok{, }\StringTok{"ARIMA+Fourier"}\NormalTok{, }\StringTok{"TBATS"}\NormalTok{,}
    \StringTok{"NN"}\NormalTok{)}

\CommentTok{\# choose model with lowest RMSE}
\NormalTok{best\_model\_index }\OtherTok{\textless{}{-}} \FunctionTok{which.min}\NormalTok{(scores[, }\StringTok{"RMSE"}\NormalTok{])}
\FunctionTok{cat}\NormalTok{(}\StringTok{"The best model by RMSE is:"}\NormalTok{, }\FunctionTok{row.names}\NormalTok{(scores[best\_model\_index,}
\NormalTok{    ]))}
\end{Highlighting}
\end{Shaded}

\begin{verbatim}
## The best model by RMSE is: NN
\end{verbatim}

\begin{Shaded}
\begin{Highlighting}[]
\FunctionTok{kbl}\NormalTok{(scores, }\AttributeTok{caption =} \StringTok{"Forecast Accuracy for Precipitation in Colombia"}\NormalTok{,}
    \AttributeTok{digits =} \FunctionTok{array}\NormalTok{(}\DecValTok{5}\NormalTok{, }\FunctionTok{ncol}\NormalTok{(scores))) }\SpecialCharTok{\%\textgreater{}\%}
    \FunctionTok{kable\_styling}\NormalTok{(}\AttributeTok{full\_width =} \ConstantTok{FALSE}\NormalTok{, }\AttributeTok{position =} \StringTok{"center"}\NormalTok{, }\AttributeTok{latex\_options =} \StringTok{"hold\_position"}\NormalTok{) }\SpecialCharTok{\%\textgreater{}\%}
    \CommentTok{\# highlight model with lowest RMSE}
\FunctionTok{kable\_styling}\NormalTok{(}\AttributeTok{latex\_options =} \StringTok{"striped"}\NormalTok{, }\AttributeTok{stripe\_index =} \FunctionTok{which.min}\NormalTok{(scores[,}
    \StringTok{"RMSE"}\NormalTok{]))}
\end{Highlighting}
\end{Shaded}

\begin{table}[!h]
\centering\centering
\caption{\label{tab:unnamed-chunk-13}Forecast Accuracy for Precipitation in Colombia}
\centering
\begin{tabular}[t]{l|r|r|r|r|r|r|r}
\hline
  & ME & RMSE & MAE & MPE & MAPE & ACF1 & Theil's U\\
\hline
ARIMA & 4.65747 & 35.41950 & 28.19275 & -0.94601 & 14.71704 & 0.06677 & 0.49434\\
\hline
STL+ETS & 5.73029 & 37.46420 & 29.10067 & -1.20640 & 15.02950 & 0.01863 & 0.47917\\
\hline
ARIMA+Fourier & 4.65747 & 35.41950 & 28.19275 & -0.94601 & 14.71704 & 0.06677 & 0.49434\\
\hline
TBATS & 12.60992 & 37.30779 & 28.73991 & 3.04859 & 14.42548 & 0.07198 & 0.53294\\
\hline
\cellcolor{gray!10}{NN} & \cellcolor{gray!10}{4.87180} & \cellcolor{gray!10}{34.49584} & \cellcolor{gray!10}{28.19237} & \cellcolor{gray!10}{-0.29821} & \cellcolor{gray!10}{14.40852} & \cellcolor{gray!10}{0.03339} & \cellcolor{gray!10}{0.51512}\\
\hline
\end{tabular}
\end{table}

\subsection{Visual Model Comparison}\label{visual-model-comparison}

\begin{Shaded}
\begin{Highlighting}[]
\FunctionTok{autoplot}\NormalTok{(test\_precipitation) }\SpecialCharTok{+} \FunctionTok{autolayer}\NormalTok{(pre\_arima, }\AttributeTok{PI =} \ConstantTok{FALSE}\NormalTok{,}
    \AttributeTok{series =} \StringTok{"ARIMA"}\NormalTok{) }\SpecialCharTok{+} \FunctionTok{autolayer}\NormalTok{(pre\_stl\_ets, }\AttributeTok{PI =} \ConstantTok{FALSE}\NormalTok{, }\AttributeTok{series =} \StringTok{"STL+ETS"}\NormalTok{) }\SpecialCharTok{+}
    \FunctionTok{autolayer}\NormalTok{(pre\_fourier, }\AttributeTok{PI =} \ConstantTok{FALSE}\NormalTok{, }\AttributeTok{series =} \StringTok{"ARIMA + Fourier"}\NormalTok{) }\SpecialCharTok{+}
    \FunctionTok{autolayer}\NormalTok{(pre\_tbats, }\AttributeTok{PI =} \ConstantTok{FALSE}\NormalTok{, }\AttributeTok{series =} \StringTok{"TBATS"}\NormalTok{) }\SpecialCharTok{+} \FunctionTok{autolayer}\NormalTok{(pre\_nnetar,}
    \AttributeTok{PI =} \ConstantTok{FALSE}\NormalTok{, }\AttributeTok{series =} \StringTok{"NN"}\NormalTok{) }\SpecialCharTok{+} \FunctionTok{xlab}\NormalTok{(}\StringTok{"Year"}\NormalTok{) }\SpecialCharTok{+} \FunctionTok{ylab}\NormalTok{(}\StringTok{"Precipitation(mm)"}\NormalTok{) }\SpecialCharTok{+}
    \FunctionTok{guides}\NormalTok{(}\AttributeTok{colour =} \FunctionTok{guide\_legend}\NormalTok{(}\AttributeTok{title =} \StringTok{"Forecast"}\NormalTok{))}
\end{Highlighting}
\end{Shaded}

\includegraphics{Colombia_code_files/figure-latex/unnamed-chunk-14-1.pdf}

\end{document}
